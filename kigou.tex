\documentclass[10pt]{article}
\usepackage{geometry}                % See geometry.pdf to learn the layout options. There are lots.
\geometry{letterpaper}                   % ... or a4paper or a5paper or ... 
%\geometry{landscape}                % Activate for for rotated page geometry
%\usepackage[parfill]{parskip}    % Activate to begin paragraphs with an empty line rather than an indent
\usepackage{graphicx}
\usepackage{amssymb}
\usepackage{epstopdf}
\usepackage{listings,jlisting}
\usepackage{color}
\usepackage{url}

\definecolor{mygreen}{rgb}{0,0,0}
\definecolor{mygray}{rgb}{0.8,0.8,0.8}
\definecolor{mymauve}{rgb}{0,0,0}

\lstset{ %
  backgroundcolor=\color{mygray},   % choose the background color; you must add \usepackage{color} or \usepackage{xcolor}
  basicstyle=\footnotesize,        % the size of the fonts that are used for the code
  breakatwhitespace=false,         % sets if automatic breaks should only happen at whitespace
  breaklines=true,                 % sets automatic line breaking
  captionpos=b,                    % sets the caption-position to bottom
  commentstyle=\color{mygreen},    % comment style
  deletekeywords={...},            % if you want to delete keywords from the given language
  escapeinside={\%*}{*)},          % if you want to add LaTeX within your code
  extendedchars=true,              % lets you use non-ASCII characters; for 8-bits encodings only, does not work with UTF-8
  frame=single,                    % adds a frame around the code
  keepspaces=true,                 % keeps spaces in text, useful for keeping indentation of code (possibly needs columns=flexible)
  keywordstyle=\color{black},       % keyword style
  language=Octave,                 % the language of the code
  morekeywords={*,...},            % if you want to add more keywords to the set
  numbers=left,                    % where to put the line-numbers; possible values are (none, left, right)
  numbersep=5pt,                   % how far the line-numbers are from the code
  numberstyle=\tiny\color{mygray}, % the style that is used for the line-numbers
  rulecolor=\color{black},         % if not set, the frame-color may be changed on line-breaks within not-black text (e.g. comments (green here))
  showspaces=false,                % show spaces everywhere adding particular underscores; it overrides 'showstringspaces'
  showstringspaces=false,          % underline spaces within strings only
  showtabs=false,                  % show tabs within strings adding particular underscores
  stepnumber=1,                    % the step between two line-numbers. If it's 1, each line will be numbered
  stringstyle=\color{mymauve},     % string literal style
  tabsize=2,                       % sets default tabsize to 2 spaces
  title=\lstname                   % show the filename of files included with \lstinputlisting; also try caption instead of title
}


\DeclareGraphicsRule{.tif}{png}{.png}{`convert #1 `dirname #1`/`basename #1 .tif`.png}

\setlength{\textwidth}{35zw}
\setlength{\textheight}{30\baselineskip}
\addtolength{\textheight}{\topskip} 

\title{ベイズ統計の理論と方法に出てくる記号まとめ}
\author{xiangze}
\date{2016.11.5}
\begin{document}
\maketitle

\section*{概要}
重要な記号のまとめです
\section*{1,2章}
\subsection*{基本的な記号}
\begin{description}

\item[ $X_i$ ] (i番目の)観測値
\item[$\Sigma_{i=1}^N$ ] 観測値に対する総和
\item[$N$ ]観測値の数
\item[$w$]モデルのパラメータ(ベクトル)
\item[$w_0$ ]モデルの中で真の分布に最も近いパラメータの値

\subsection*{分布と関数}
\item[$q(x)$] 真の分布
\item[$p(X|w)$] 統計モデル
\item[$\phi(w)$] 事前分布
\item{$E_X \bigl[ f(x) \bigr] \equiv \int f(x)q(x)dx$} 真の分布による平均
\item{$E_w \bigl[ f(w) \bigr] \equiv \int f(w)p(w|X^n)dw $} 事前分布による平均
\item[$p(w|X^n) \propto p(X^n|w)\phi(w)$]
\item[$p(w|X^n) = \frac{p(X^n|w)\phi(w)}{Z}$]
ここでは $ Z= \int \prod_{i} p(X_i|w) \phi(w) dw$ 分配関数

\item[$ p(w|X^n) = \frac{\prod_n p(X^n|w)\phi(w)}{Z_n(\beta)}$]
\item[$ Z_n(\beta)= \int \prod_{i} p(X_i|w)^\beta \phi(w) dw$] 分配関数
\item[$F_n(\beta)=-\frac{1}{\beta}\log Z_n(\beta)$] 自由エネルギー

\subsection*{損失関数}
\item{$G_n \equiv -E_X \bigl[ \log p^*(x) \bigr] =-E_X\bigl[ \log E_w p(X|w) \bigr] $} 汎化損失 
\item{$T_n \equiv -\frac{1}{n}\sum_i \log p^*(X_i) =-\frac{1}{n}\sum_i \log E_w p(X|w) $} 経験損失 ($p^*(x)$は予測分布)
\item{$L(w)=-E_X \bigl[ \log p(X|w) \bigr] $} 
\item{$-\int q(x)\log q(x)dx+\int q(x)\log\frac{p(x)}{p(x|w)} dx $}平均対数損失
\item{$f(x,w)=f(x|w,w_0)=\log{p(x|w_0)}{p(x|w)}$} 対数尤度比関数 また$p(x|w)=p_0(x)e^{-f(x)}$ とも書く
\item{$K_n(w)=\frac{1}{n}\sum_i f(X_i,w)$}経験誤差
\item{$L_n(w)=-\frac{1}{n}\sum_i \log p(X_i,w)$}

\end{description}

\section*{3章 正則理論}

\section*{4章 一般理論}

\section*{link}
http://watanabe-www.math.dis.titech.ac.jp/users/swatanab/bayes-theory-method.html

\end{document}  
